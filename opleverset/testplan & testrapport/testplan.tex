\documentclass{article}
\usepackage[margin=1in]{geometry}
\usepackage{graphicx} 
\usepackage[dutch]{babel}
\usepackage[scaled]{helvet}
\renewcommand\familydefault{\sfdefault} 
\usepackage[T1]{fontenc}
\usepackage{datetime}
\newdateformat{monthyeardate}{%
  \monthname[\THEMONTH], \THEYEAR}

\begin{document}
\sffamily
\begin{titlepage}
  \centering
    \vfill
    \vfill
    \includegraphics[width=4cm]{logohr} % also works with logo.pdf
    \vfill
    \vfill
    {\bfseries\Huge
      Testplan\\
      Project 2 ACM - TINPRJ04-2
      \vskip6cm
    }
      {
        \bfseries\normalsize
        \textbf{\underline{Student:}}
        \hfill
        \textbf{\underline{Vakdocent:}} \\
        Maurice Bal - 1032062
        \hfill
        Daisy Hofman\\
        Prashant Chotkan - 1042569
        \hfill
        Thijs de Ruiter\\
        \vskip6cm       
      }
      {\bfseries\Large
        \textit{Rotterdam University of Applied Sciences}\\
        \monthyeardate\today\\
      }
    \vfill
    
    \vfill
    \vfill
\end{titlepage}
\newpage
\tableofcontents
\newpage

\section{Sensor: InfraRood}
\subsection{Test 1.1: Laat de ACM op de lijn afrijden zodat de ACM, de lijn in de lengte zal tegenkomen.}
\subsubsection{Testopstelling}
Een zwarte lijn met een breedte van ong. 3cm en een lengte van ong. 10cm geplaatst in een open ruimte. \\
\begin{center}
\includegraphics[width=10cm]{test1.1/test1.1}
\end{center}
\subsubsection{resultaten}
\textit{Verwacht resultaat: } De IR sensor zal de lijn detecteren en de ACM zal tot stilstand komen.
\newline
\textit{Acceptance Criteria: }De ACM moet tot stilstand komen. De ACM mag niet met de wielen op de lijn staan.
\newline
\textit{Waarnemingen: }De ACM stopt voordat de wielen de lijn geraakt hebben
\newline
\textit{Conclusie: }De test is geslaagd

\section{Sensor: Ultrasoon}
\subsection{Test 2.1: Laat de ACM recht op de muur afrijden.}
\subsubsection{Testopstelling}
Een muur met een lengte en hoogte van minimaal 20cm.\\
INSERT FOTO
\subsubsection{resultaten}
\textit{Verwacht resultaat: }De Ultrasson sensor zal de muur detecteren en de ACM zal tot stilstand komen.
\newline
\textit{Acceptance Criteria: }De ACM moet tot stilstand komen en de ACM moet op een afstand van minimaal 3cm van de muur staan.
\newline
\textit{Waarnemingen: }
\newline
\textit{Conclusie: }

\section{Sensor: LDR}
\subsection{Test 3.1: Laat de ACM door de opening van de tunnel rijden}
\subsubsection{Testopstelling} 
INSERT FOTO
\subsubsection{resultaten}
\textit{Verwacht resultaat: }De LDR zal de verandering in licht door de tunnel detecteren en de ACM zal tot stilstand komen.
\newline
\textit{Acceptance Criteria: }De ACM moet tot stilstand komen en de ACM mag tot maximaal 50\% met zijn frame in de lengte in de tunnel terecht komen.
\newline
\textit{Waarnemingen: }
\newline
\textit{Conclusie: }

\section{Sensor: Reed}
\subsection{Test 4.1: Laat de ACM op de magneetstrip afrijden zodat hij deze in de lengte tegen zal komen}
\subsubsection{Testopstelling}
Een magneetstrip met een breedte van ong. 2cm en een lengte van ong. 10cm geplaatst in een open ruimte.\\
INSERT FOTO
\subsubsection{resultaten}
\textit{Verwacht resultaat: }De reed sensor zal de magneetstrip detecteren en ACM zal tot stilstand komen.
\newline
\textit{Acceptance Criteria: }De ACM moet tot stilstand komen.
\newline
\textit{Waarnemingen: }
\newline
\textit{Conclusie: }

\newpage




\section{Feature 1: De ACM kan door een tunnel rijden}

\subsection{Test 5.1: Laat de ACM recht op de muur afrijden}
\subsubsection{Testopstelling}
Een muur met een hoogte van minimaal 20cm en een lengte van minimaal 40cm\\
INSERT FOTO
\subsubsection{resultaten}
\textit{Verwacht resultaat: }De ACM zal de muur detecteren en langs de muur door blijven rijden
\newline
\textit{Acceptance Criteria: }De ACM kan langs de muur rijden zonder deze te raken op een afstand van minimaal 1cm. 
\newline
\textit{Waarnemingen: }
\newline
\textit{Conclusie: }

\subsection{Test 5.2: Laat de ACM door de opstelling rijden}
\subsubsection{Testopstelling}
Een tunnel met een opening van 30cm bij 30cm en een lengte van ong. 40cm\\
INSERT FOTO
\subsubsection{resultaten}
\textit{Verwacht resultaat: }De ACM zal de tunnel detecteren en door de tunnel rijden
\newline
\textit{Acceptance Criteria: }De ACM kan door de tunnel rijden zonder de muren van de tunnel te raken op een afstand van minimaal 1cm tussen de ACM en de muur. 
\newline
\textit{Waarnemingen: }
\newline
\textit{Conclusie: }


\section{Feature 2: De ACM moet op een accu rijden}
\subsection{test 6.1: Laat de ACM rijden autonoom op enkel de stroom van een accu rijden}
\subsubsection{Testopstelling}
Een baan zoals op de foto te zien is. De zwarte lijnen liggen 40cm uit elkaar en de lijnen zelf zijn 3cm breed\\
INSERT FOTO
\subsubsection{resultaten}
\textit{Verwacht resultaat: } De ACM zal de baan volgen en blijven rijden
\newline
\textit{Acceptance Criteria: }De ACM kan minimaal 5 minuten de baan blijven volgen   
\newline
\textit{Waarnemingen: }
\newline
\textit{Conclusie: }

\section{Feature 3: De ACM kan in een parkeervak op het haventerrein parkeren}
\subsection{Test 7.1: Parkeer de ACM m.b.v. de remote controller in het parkeervak}
\subsubsection{Testopstelling}
Een parkeervak van 40cm bij 30cm. De zwarte lijnen hebben een breedte van 30mm\\
\begin{center}
  \includegraphics[width=10cm]{test7.1/test7.1}
\end{center}
\subsubsection{resultaten}
\textit{Verwacht resultaat: }De ACM zal reageren op de controller en in het parkeervak geplaats worden
\newline
\textit{Acceptance Criteria: }De ACM moet zodanig in het parkeervak geplaatst kunnen worden zodat de lijnen van het parkeervak niet geraakt worden.
\newline
\textit{Waarnemingen: }De ACM is in het parkeervak tot stilstand gekomen zonder de lijnen van het parkeervak te raken
\newline
\textit{Conclusie: }De test is geslaagd

\section{Feature 4: De ACM kan over heuvels rijden}

\subsection{Test 8.1: Laat de ACM over een helling rijden}
\subsubsection{Testopstelling}

INSERT FOTO
\subsubsection{resultaten}
\textit{Verwacht resultaat: }De ACM kan zonder te stoppen over de heuvel rijden.
\newline
\textit{Acceptance Criteria: }De ACM moet over de heuvel kunnen rijden en mag op maximaal de binnenste 15mm van de zwarte lijnen van het rijvlak komen.
\newline
\textit{Waarnemingen: }
\newline
\textit{Conclusie: }

\section{Feature 5: De ACM kan binnen vijf minuten de containers naar de bestemming op het haventerrein brengen}
\subsection{Test 9.1 Laat de ACM de testopstelling volgen}
\subsubsection{Testopstelling}
Er wordt een baan afgelegd zoals weergegeven in de foto. Het vak is in totaal 3 bij 3 meter en de zwarte lijnen zijn 3cm breed. 
INSERT FOTO
\subsubsection{resultaten}
\textit{Verwacht resultaat: }De ACM zal de opgegeven baan volgen
\newline
\textit{Acceptance Criteria: }De ACM moet de volledige baan binnen 5 minuten kunnen afleggen
\newline
\textit{Waarnemingen: }
\newline
\textit{Conclusie: }

\section{Feature 6: De ACM kan na een val van maximaal 50 mm verder rijden}
\subsection{Test 10.1 Laat de ACM van een hoogte van 5cm vallen}
\subsubsection{Testopstelling}

INSERT FOTO
\subsubsection{resultaten}
\textit{Verwacht resultaat: }De ACM zal door blijven rijden 
\newline
\textit{Acceptance Criteria: }De ACM moet na de val zonder externe hulp verder kunnen rijden
\newline
\textit{Waarnemingen: }
\newline
\textit{Conclusie: }

\section{Feature 7: De ACM kan binnen de lijnen van het rijvlak rijden}
\subsection{Test 11.1 Laat de ACM op een lijn af rijden}
\subsubsection{Testopstelling}
Een zwarte lijn van 3cm breed met een lengte van ong. 30cm \\
\begin{center}
  \includegraphics[width=10cm]{test11.1/test11.1}
\end{center}
\subsubsection{resultaten}
\textit{Verwacht resultaat: }De ACM zal de lijn detecteren en langs de lijn rijden
\newline
\textit{Acceptance Criteria: }De ACM kan langs de lijn rijden en mag op de lijn komen, maar mag daar met geen enkel onderdeel aan de andere kant uitsteken.
\newline
\textit{Waarnemingen: }De ACM volgt de lijn zonder deze aan te raken
\newline
\textit{Conclusie: }De test is geslaagd

\subsection{Test 11.2 Laat de ACM de testopstelling volgen}
\subsubsection{Testopstelling}
een baan met twee bochten, de lijnen liggen 40cm uitelkaar en de lijnen zijn 3cm breed.
INSERT FOTO
\subsubsection{resultaten}
\textit{Verwacht resultaat: }De ACM zal de baan volgen zonder de lijnen van het rijvlak te overschrijden
\newline
\textit{Acceptance Criteria: }De ACM kan de baan volledig volgen en mag op de lijn komen, maar mag daar met geen enkel onderdeel aan de andere kant uitsteken.
\newline
\textit{Waarnemingen: }
\newline
\textit{Conclusie: }

\section{Feature 8: De ACM kan obstakels ontwijken}
\subsection{Test 12.1 ....}
\subsubsection{Testopstelling}
\subsubsection{resultaten}
\textit{Verwacht resultaat: } ........
\newline
\textit{Acceptance Criteria: }
\newline
\textit{Waarnemingen: }
\newline
\textit{Conclusie: }

\section{Feature 9: De ACM kan bij de kade(afgrond) keren}
\subsection{Test 13.1 Laat de ACM op een afgrond afrijden}
\subsubsection{Testopstelling}
\subsubsection{resultaten}
\textit{Verwacht resultaat: }De ACM zal de afgrond detecteren, 180 graden draaien en weer terug rijden.
\newline
\textit{Acceptance Criteria: }De ACM mag niet van de afgrond vallen, en moet 180 graden gedraaid zijn, voordat deze weer verder rijdt
\newline
\textit{Waarnemingen: }
\newline
\textit{Conclusie: }

\section{Feature 10: De ACM kan 2 containers van 60 bij 30 bij 30 mm vervoeren}

\subsection{Test 14.1 Laat de ACM met twee containers van 60 bij 30 bij 30 mm over een baan rijden}
\subsubsection{Testopstelling}
baan met helling ... \\
INSERT FOTO
\subsubsection{resultaten}
\textit{Verwacht resultaat: }De ACM kan met de containers over de baan rijden zonder dat deze eraf vallen.
\newline
\textit{Acceptance Criteria: }De ACM kan de volledige baan afleggen, zonder de containers te laten vallen
\newline
\textit{Waarnemingen: }
\newline
\textit{Conclusie: }

\section{Feature 11: De ACM kan magnetische punten op het terrein detecteren als herkeningspunten}
\subsection{Zie test 4.1}



\newpage
\section{Deel 1}
Een stukje tekst\ldots 
\section{Sectie 2}
\subsection{Opgave 1}
\subsection{Opgave 2}
\subsubsection{nog dieper}
\subsubsection{nog dieper}
\subsubsection{nog dieper}
\subsubsection{nog dieper}
\subsection{Opgave 3}
\section{Deel 2}
\section{Deel 3}
Een stukje tekst\ldots \\
op een nieuwe regel\ldots 
\section{Enkele voorbeelden}
Een formule in $f(x)=3x^2+7$ een zin, zoals $\frac{1}{\pi}$ of $\sqrt{e^{2}}$ is mogelijk\\
Een formule in een aparte environment:
 \begin{eqnarray}
   \sum_{1}^{n}n=\frac{1}{2}n\cdot(n+1)\\
   \prod_{1}^{n}n=n!\\
   f(x)=5x^3+\sqrt{2x}
 \end{eqnarray}
 
Een plaatje in jpg, png of pdf werkt als volgt:
 \begin{center}
\includegraphics[width=2cm]{logohr}
\end{center}
\section{Opsommingen}
\begin{itemize}
\item kaas
\item melk
  \begin{enumerate}
\item rood
\item geel
  \begin{itemize}
  \item groen
  \item paars
  \item oranje
  \item $e^x+6$
  \end{itemize}
\item blauw
\end{enumerate}
\item eieren
\end{itemize}
Ten slotte een tabel met de tabular environment:\\
\vspace{5mm}

\begin{tabular}{|l|r||c|}
  \hline
  links &rechts &midden\\
  \hline\hline
  blabla&bleble&5345\\
  \hline
  blabla&\ldots bleble\ldots &5345\\
  \hline
  \ldots blabla\ldots &bleble&24245345\\
  \hline
  blabla&bleble&534222342345\\
  \hline
\end{tabular}
\end{document}









